\chapter{Introducción}

Nuestra sociedad esta tanto rodeada como conformada por sistemas, desde todos los que constituyen nuestro cuerpo hasta los vehiculos o las redes de trenes. 

Presentes en la era informacional, nos interesa estudiar los sistemas de información por el ritmo, organización y modo de trabajo de nuestra sociedad. Estos son un conjunto de elementos orientados con el fin de cubrir una necesidad u objetivo que interactuan entre si para manejar entrada, procesamiento, almacenamiento y salida de datos e información.

La presente es una investigación monográfica donde serán abordados diferentes tipos de sistemas de información tales como: 
\begin{itemize}
\item Transaction Processing System (TPS)
\item Management Information System (MIS)
\item Executive Support System (ESS) 
\item Desision Support System (DSS) 
\end{itemize}

Ésto en miras de conocer de que consiste cada uno de ellos, conocer sus características, proposito, estructura, el tipo de información que manejan, organizaciones y areas de éstan en los cuales son típicamente utilizados asi como beneficios, desventajas y algunos casos de estudio.

A continuación se desarrollan dichos aspectos en un capítulo para cada sistema.

