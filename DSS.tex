\chapter[DSS]{DSS Decision Support System}
\section {Definición}
Existen varias definiciones que parten del mismo punto, es un sistema diseñado para asistir la toma de decisiones. Según Kenn “no puede haber una definición de los sistemas de apoyo a la decisión, sino sólo del apoyo a la decisión” [16].

Los DSS son muy diversos, buscando modelar los problemas y dar resultados más rápidos o más precisos, dependiendo de las exigencias del usuario.

\subsection{Características}
\begin{itemize}
\item Flexibilidad en el uso de la información.
\item Facilidad para el volcado de datos.
\item Comúnmente llevan imbuidas herramientas de modelado y simulación.
\item Combinar información interna y externa a la empresa.
\item Permite análisis multidimensional.
\item Proyecciones y predicciones.
\item Poco requerimiento de conocimientos técnicos.
\item Rapidez en el tiempo de respuesta.
\item Historial.
\end{itemize}

\subsection {Clasificación:}

Según Haettenschwiler, usando su clasificación por la relación con el usuario, se pueden diferenciar los DSS de la siguiente forma:
\begin{description}
\item [DSS pasivo:] Ayuda en el proceso de decisiones mas por sí solo no está en capacidad de tomar decisiones ni de dar sugerencias.
\item [DSS activo:] Puede tomar decisiones y sugerencias.
\item [DSS cooperativo:] Genera soluciones y sugerencias que pueden ser modificadas por el personal encargado de la toma de decisiones y así se va sofisticando iterativamente entre el sistema y el personal hasta llegar a una única decisión.
\end{description}

Según Power, en su clasificación por modo de asistencia, se ven:
\begin{description}
\item [DSS dirigidos por modelos:] Se basan en modelos estadísticos y de simulación, entre otros.
\item [DSS dirigidos por comunicación:] Soportan multisesión para el trabajo colectivo sobre la misma tarea.
\item [DSS dirigidos por datos:] Dan prioridad a la manipulación de los datos que tienen relevancia en la decisión a tomar.
\item [DSS dirigidos por documentos:] Gestionan información archivada en distintos tipos de achivos.
\item [DSS dirigidos por conocimientos:] Utilizan experiencias previas para tomar decisiones.
\end{description}

También según Power, en su clasificación por ámbito, se tiene:
\begin{description}
\item [DDS para la gran empresa:] Brinda servicio a mucho personal.
\item [DSS de escritorio:] Brinda servicio a una sola persona.
\end{description}

\section {Propósito}
Proveer servicios relacionados con el análisis de datos, asistir en el proceso de toma de decisiones mediante procedimientos y modelos.

\section {Estructura / Elementos / Componentes}
\begin{description}
\item [Interfaz amigable para facilitar el volcado de datos]
Una interfaz sencilla que permita el ingreso y manipulación simple de datos por parte de los usuarios.
\item [Informes dinámicos, flexibles e interactivos]
Un sistema que refleja las necesidades de los usuarios de tal forma que los informes las satisfagan.
\item [Rapidez en el tiempo de respuesta]
Debido a la gran cantidad de datos manejados en las empresas, la optimización de procesamiento suele ser de gran importancia.
\item [Manejo de perfiles]
Cada usuario mantiene un perfil y se le muestra información adecuada a sus responsabilidades.
\item [Sistema manejador de Bases de Datos]
El sistema de almacenamiento de los datos a analizar.
\item [Sistema gestor de Modelos]
Un repositorio donde se almacenen los modelos matemáticos y los procedimientos a ejecutar por el sistema en un análisis.
\item [Los Usuarios]
en concordancia con Marakas, los usuarios son parte del sistema, en mi opinión es una parte fundamental. El sistema es usado por ellos, hecho por ellos y hecho para ellos.
\end{description}

\section {Información que manejan}
Dependiendo del área de trabajo del usuario o del tipo de soporte que ofrece el sistema, este puede llegar a manejar más o menos información. En la práctica es el usuario quien decide el tipo de información que maneja el sistema sin embargo en todas las versiones y aplicaciones se usan grandes cantidades de datos para procesar.

De todas las posibilidades resaltan las siguientes:
\begin{itemize}
\item Sistemas de información gerencial (MIS)
\item Sistemas de información ejecutiva (EIS)
\item Sistemas expertos basados en inteligencia artificial (SSEE)
\item Sistemas de apoyo a decisiones de grupo (GDSS)
\end{itemize}

\section {Organizaciones que típicamente utilizan DSS}
\begin{itemize}
\item Microsoft a través de su producto Microsoft NetMeeting. Este puede ser considerado un DSS dirigido por comunicación.
\item Microsoft SharePoint Workspace es otro ejemplo de DSS dirigido por comunicación.
\item En general cualquier persona o empresa que utilice las bondades actuales  de la llamada nube informática para algún beneficio empresarial, está  usando DSS, estos beneficios van desde videoconferencias hasta programas especializados en el análisis de datos.
\end{itemize}

\section {Áreas o funciones que apoyan en las organizaciones}
Repositorios de archivos, servidores de datos, análisis y transferencias de datos, video conferencias, son algunos de los servicios que los DSS brindan en las diferentes empresas y que giran en torno a una meta, asistencia en la toma de decisiones.

\section {Beneficios y Desventajas}

\subsection{Beneficios}
\begin{description}
\item [Informes dinámicos, flexibles e interactivos:] permite que los reportes sean generados de forma provechosa para el usuario, ajustado a sus necesidades y exigencias.
\item [No requiere conocimientos técnicos:] permite que el usuario pueda usar los beneficios de los sistemas con un bajo nivel de adiestramiento por parte la empresa.
\item [Integración entre los diferentes departamentos:] permite que la información que se almacena en la base de datos sea lo menos ambigua posible permite que se integren así las diferentes divisiones de la empresa.
\item [Disponibilidad de información histórica:] permite consultas a los datos de otros períodos y hacer estudios históricos.
\end{description}

\subsection{Desventajas}
\begin{description}
\item Necesidad de mayor conocimiento de uso por parte del encargado de la toma dedecisiones.
\end{description}

%\section {Ejemplos/ Casos de estudio}
