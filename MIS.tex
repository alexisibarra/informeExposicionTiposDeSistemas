\chapter[MIS]{MIS Management Information System}
\section {Definición}
Es un sistema basado en las interacciones hombre – computadora usado comúnmente a nivel empresarial para convertir el flujo de datos que manejan en información útil que describa a la compañía lo ocurrido, lo que está sucediendo en el presente y mediante un amplio análisis probabilístico y estadístico, lo que podría suceder en el futuro; permitiendo tomar las decisiones necesarias y adecuadas para resolver los problemas de la compañía.
\section {Propósito}
Satisfacer las necesidades que tienen los gerentes o subunidades de una compañía para el manejo de información general. Ofrecen la oportunidad a las compañías de lograr un control eficaz sobre la gran cantidad de datos que manejan para filtrarla y omitir información irrelevante y redundante. De esta manera le agrega un valor a los datos y un significado para convertirlos en la información necesaria para que el gerente pueda realizar las decisiones adecuadas.
\section {Estructura / Elementos / Componentes}
El sistema requiere de una base datos que almacene los datos e información del entorno suministrados por el TPS. Adicionalmente necesita de un software orientado a dos campos: generador de informes periódicos y especiales; y un modelo de simulación matemático que realice el análisis probabilístico y estadístico necesario, utilizando los datos de la base de datos, para generar los posibles escenarios a ocurrir.

Principalmente requiere de una base humana, un personal capacitado que pueda analizar toda la información que genera el sistema y así realizar la toma de decisiones debida para solucionar los problemas de la compañía y mantenerla estable dentro del mercado.
\section {Información que manejan}
La información que maneja el sistema se encuentra agrupado en base a  una estructura piramidal que identifica la importancia y prioridad de los datos, para que éstos sean distribuidos en el lugar correspondiente dentro de la compañía.

En el  primer lugar de la pirámide, se debe conocer detalladamente el estado y veracidad de los datos, es decir, analizar los datos referidos al procesamiento de las transacciones.

En segundo lugar, la evaluación de los datos para conocer los recursos de información, que permitirá ejecutar y controlar las operaciones diarias de la empresa. 

En el tercer nivel se ubican todos los recursos que posee el sistema de información, que se encargan de ofrecer los datos requeridos para lograr la planificación y así establecer la toma de decisiones necesaria en la administración de la compañía. 

Finalmente, en el último nivel de la pirámide encontramos los recursos de información que dejarán la oportunidad de planificar las estratégicas correspondientes y las posibles políticas a implementar en los niveles más altos de la administración.

\section {Organizaciones que típicamente utilizan TPS}
Debido a la relevancia que posee la información real y a tiempo en las empresas, tanto grandes compañías como las Pymes, éstas integran el MIS y responden al mercado en forma rápida y creativa ya que el sistema les brinda el apoyo necesario para la toma de decisiones y les otorga la oportunidad de competir y crecer en su rama. Por todo esto el sistema un factor determinante en el desarrollo de la compañía.

\section {Áreas o funciones que apoyan en las organizaciones}
\begin{itemize}
\item Identificación y comprensión de los problemas.
\item Planificación de estrategias y políticas en base a las decisiones tomadas.
\item Organización de los datos e información.
\item Control dentro de las subunidades de la compañía.
\end{itemize}
\section {Beneficios y Desventajas}
\subsection{Beneficios}
\begin{itemize}
\item Posee la capacidad de manejar un gran flujo de datos y tratarlos según los lineamientos necesarios para convertirlos en información útil para la compañía
\item Las simulaciones obtenidas por la información del sistema permiten considerar múltiples alternativas que confieren la capacidad de evaluar el impacto de las decisiones y circunstancias en los que esté involucrado la compañía.
\end{itemize}
\subsection{Desventajas}
\begin{itemize}
\item Requiere un alto nivel de conocimiento matemático para entender y desarrollar un modelo adecuado para el manejo de la información otorgada por el sistema.
\item El proceso de integración del sistema a la compañía y adicionalmente en la búsqueda de rapidez y eficiencia, involucraría un nivel de costo económico alto para la compañía.
\end{itemize}
\section {Ejemplos/ Casos de estudio}
\subsection{Polar}
\begin{itemize}
\item Empresas Polar se encarga de la producción, distribución y venta de una amplia gama de alimentos y bebidas a nivel nacional e internacional; adicionalmente de su aporte en servicios que le brinda a la comunidad venezolana. Para poder cumplir con estas funciones emplean un sistema de gestión empresarial que genera todas las estadísticas necesarias para que los gerentes y demás empleados capacitados conozcan el estado de la empresa y tomen decisiones en base a eso. En el ámbito también se le conoce con el nombre \emph{Business Intelligent}.
\end{itemize}
\subsection{DirecTV}
\begin{itemize} 
\item Directv es una empresa suscriptora de televisión por satélite, la cual maneja un amplio flujo de datos de suscripciones, programaciones, facturación y servicios que presta a sus usuarios. Todos estos datos los procesa por el sistema de información gerencial, el cual, por departamentos, permite tener un control automatizado de sus diversos servicios a sus suscriptores.
\end{itemize}
