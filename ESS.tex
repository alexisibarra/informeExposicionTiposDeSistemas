\chapter[ESS]{ESS Executive Support System}
\section {Definición}
La traducción directa de las siglas ESS (Executive Support System) es “Sistemas de Soporte Ejecutivo”. Estos sistemas surgen de la necesidad de controlar la manera en que las decisiones en medianas y grandes empresas son tomadas, puesto que anteriormente las decisiones eran tomadas por los altos ejecutivos, y el peso de las mismas estaban prácticamente basadas en la experiencia o habilidad de estas personas en las respectivas áreas o cargos que ocupaban, o en su defecto el numero de acciones que estos tenían en la empresa.

Esta manera de toma de decisiones permitía desviar a intereses particulares el rumbo de una empresa y con esto se generaban grandísimas probabilidades de poner en riesgo el éxito de la misma.

\begin{quote}
“Los Sistemas de Soporte Ejecutivos son una herramienta (software) de reportes, que permite convertir datos organizativos en útiles informes resumidos. Estos reportes generalmente serán usados por gerentes, administradores y jefes para un rápido acceso a informes que podrían ser generados por todos los estratos de la empresa y/o departamentos. 
Aunado a esto también podría proporcionar herramientas de análisis que predice una serie de resultados de desempeño en el tiempo utilizando los datos de entrada” [4]
\end{quote}
\subsection{Características de los ESS} 
\begin{itemize}
\item Acortan notablemente el tiempo de obtención de resultados con respecto a procedimientos manuales sin perder el nivel de consistencia de los mismos.
\item Son sistemas ajustados a las necesidades específicas de la empresa.
\item Suministran información interactiva e ilustrativa para la fácil interpretación de resultados y eficiente toma de decisiones. (Texto, Tablas, Gráficos, Audio, Video, etc.)
\item Toda la información y resultados deben ser fácilmente accesibles en línea (medio común para los ejecutivos). Con rápida respuesta de los servicios.
\item Debe ser fácil / intuitivo de usar con una interfaz amigable para el usuario.
\end{itemize}
\section {Propósito}
El propósito de los ESS se ve muy bien descrito en un segmento de “Sistema de Apoyo a Ejecutivos – Funciones no comerciales” de Alma López:

\begin{quote} 
“Los sistemas de apoyo a ejecutivos, son creados específicamente a la alta dirección, posicionándose en un factor clave, debido a que buscan que se pueda observar, monitorear y dar seguimiento a los factores críticos para el éxito de la organización. Entre lazan información interna y externa de la empresa, planteando un panorama completo para la toma correcta de decisiones.
  
Para conducir una empresa al éxito, no importando su naturaleza, se debe estar saturado de la información necesaria, es decir, se debe conocer, comprender, analizar nuestros pro y contra, hacer énfasis en nuestras áreas fuertes y mejorar las débiles. Es ahí donde la información proporcionada o suministrada, juega un papel importante debido a que un sistema que apoye directamente a los altos ejecutivos, le permitirá mejorar su panorama para dirigir a la organización” [5] 
\end{quote}

\section {Estructura / Elementos / Componentes}
\begin{itemize}
\item Una interfaz con el usuario que permita una cómoda y eficiente comunicación con el equipo que contiene el software, donde se puedan registrar las informaciones pertinentes para el análisis, y reportes.
\item Un servidor que pueda manejar la información del sistema, este servidor debe estar en la capacidad de suministrar información a la base de datos, como también devolver información obtenida de la base de datos.
\item Una base de datos donde se almacena  de manera organizada la información del sistema.
\item Motores gráficos que elaboren las gráficas resultantes del análisis de la herramienta.
\item Analizadores estadísticos para la producción de resúmenes cuantitativos y cualitativos para el criterio de decisión.
\end{itemize}

\section {Información que manejan}
Esencialmente los datos que manejan los ESS son muy variados, puesto que engloban toda la información necesaria para tomar decisiones óptimas. En el ámbito empresarial, absolutamente todo es relevante, desde datos cuantitativos como:

\begin{itemize}
\item Facturación
\item Contabilidad de costos
\item Nomina
\item Programación
\item Entre otros
\end{itemize}

Como también datos cualitativos o descriptivos que se puedan considerar de peso para la elaboración de criterios de toma de decisiones:

\begin{itemize}
\item Descripciones de procesos
\item Moral del personal
\item Ambiente de trabajo
\item Inteligencia emocional
\item Resistencia al cambio
\item Algunos otros aspectos que entran en el ámbito del coaching. 
\end{itemize}

\section {Organizaciones que típicamente utilizan ESS}
Empresas de diferentes ámbitos:
\begin{description}
\item [Financieras:] Los directivos del Royal Bank of Canada, Utilizan ESS para la selección de sus criterios y toma de decisiones.[6]
\item [Medicina:] MEDITECH proporciona soluciones de software integradas que satisfacen las necesidades de las organizaciones del cuidado de la salud alrededor del mundo. Las organizaciones a las que se provee servicio incluyen hospitales, centros de cuidado ambulatorio, consultorios médicos, organizaciones de cuidado en el hogar e instalaciones de cuidado a largo plazo y de salud mental.[7]
\item [Alimentos y Bebidas:] La Bodega Sutter Home utiliza los datos externos, sobre todo incluyendo información del Internet, en su EES. Para la óptima toma de decisiones con respecto a producción y mercado.[6]
\item [Transporte:] Cambridge Systematics (procesos de planificación, ejecución y evaluación para transportes y vías de transporte)[8]
\end{description}

\section {Áreas o funciones que apoyan en las organizaciones}
\begin{description}
\item [Royal Bank of Canadá:] Los altos ejecutivos son capaces de escoger sus propios criterios de entre múltiples opciones, mediante una interfaz fácil de usar. Logrando así un rápido manejo de cuestiones de inversión, estadística y mercado.[6]
\item [Meditech:] Obtención de criterios de inversión, prioridades de desarrollo investigativo, coordinar el cuidado de la salud de mejor manera, prevenir errores médicos, agilizar flujos de trabajo, apresurar los reembolsos e ingresos y operar más eficientemente.[7]
\item [Bodega Sutter Home:] Es el instrumento principal en la elección de criterios para la toma de decisiones en la empresa en casi todo ámbito.[6]
\item [Transporte:] Pilar en la elaboración de criterios ejecutivos para la aprobación de actividades como elaboración de estrategias y manejo del transito, planeacion de rieles para trenes, análisis económicos, transportación segura, entre otros.[8]
\end{description}

\section {Beneficios y Desventajas}

\subsection{Beneficios}
\begin{description}
\item [Royal Bank of Canadá:] Los altos ejecutivos son capaces de escoger sus propios criterios de entre múltiples opciones, mediante una interfaz fácil de usar. Logrando así un rápido manejo de cuestiones de inversión, estadística y mercado.[6]
\item [Meditech:] Obtención de criterios de inversión, prioridades de desarrollo investigativo, coordinar el cuidado de la salud de mejor manera, prevenir errores médicos, agilizar flujos de trabajo, apresurar los reembolsos e ingresos y operar más eficientemente.[7]
\item [Bodega Sutter Home:] Es el instrumento principal en la elección de criterios para la toma de decisiones en la empresa en casi todo ámbito.[6]
\item [Transporte:] Pilar en la elaboración de criterios ejecutivos para la aprobación de actividades como elaboración de estrategias y manejo del transito, planeacion de rieles para trenes, análisis económicos, transportación segura, entre otros.[8]
\end{description}

\subsection{Desventajas}
\begin{itemize}
\item Los sistemas de Soporte a Ejecutivos no proveen una decisión sintetizada óptima, solo los criterios de evaluación. 
\item Necesitan muchísima información para poder dar resultados fiables.
\item Se requiere tiempo de preparación y análisis para obtener la información deseada 
\item Dificultad para mantener la integridad de la base de datos 
\end{itemize}

\section {Ejemplos/ Casos de estudio}
\subsection {Meditech} 
Proporciona soluciones de software integradas que satisfacen las necesidades de las organizaciones del cuidado de la salud alrededor del mundo. Para esto utilizan sistemas ESS, para la elaboración de criterios de decisión que faciliten el tomar el mejor sendero de éxito en el asesoramiento de otras organizaciones.

Las organizaciones a las que MEDITECH proporciona asesoramiento incluyen hospitales, centros de cuidado ambulatorio, consultorios médicos, organizaciones de cuidado en el hogar e instalaciones de cuidado a largo plazo y de salud mental.

\subsection {Cambridge Systematics:} Son especialistas en transporte, dedicados a asegurar que las inversiones en transporte generen los mejores resultados posibles.

Esta empresa utiliza los sistemas ESS para generan sus criterios de decisión que luego de ser refinados y adaptados al contexto de las necesidades de sus clientes tendrán como resultado políticas innovadoras y soluciones de planificación, el análisis objetivo, y aplicaciones de la tecnología.

Proveen soporte a sus clientes para satisfacer las necesidades futuras de transporte al mismo tiempo que mejoran el rendimiento de las operaciones e infraestructura existente.
